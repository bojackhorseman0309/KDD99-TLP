\documentclass[12pt,a4paper]{article}

\usepackage[utf8]{inputenc}
\usepackage[spanish]{babel}
\usepackage{palatino}
\usepackage{amsmath}
\usepackage{amssymb}
\usepackage{url}
\usepackage{hyperref}
\usepackage{enumerate}
\usepackage{geometry}
\usepackage{booktabs}
\usepackage{underscore}
\usepackage{longtable}
\usepackage{graphicx}
\usepackage{listings}

\geometry{
  a4paper,
  left=2.5cm,
  right=2.5cm,
  top=3cm,
  bottom=3cm
}

\begin{document}

\title{Análisis de Datos para Ciberseguridad: Trabajo práctico 2}

\author{
  Alonso Araya Calvo \\
  Pedro Soto \\
  Sofia Oviedo \\
  Instituto Tecnológico de Costa Rica, \\
  Escuela de Ingeniería en Computación, \\
  Programa de Maestría en Ciberseguridad
}

\date{ 14 de setiembre de 2025 }
\maketitle

En este trabajo practico se realiza un análisis y implementación de un árbol de decisión para el set de datos KDD99.

Para efectos del proyecto se tomo un dataset reducido de KDD99, el cual contiene 10\% de los datos originales,
en el cual se va tomar en cuenta solamente las clases `normal' que indica tráfico normal y `backdoor' que indica tráfico malicioso.
Además de eso se creo una versión filtrada del dataset, sin duplicados, filas sin valores y datos categóricos convertidos a
numéricos mediante one-hot-encoding como lo son `protocol type', `service' y `flag'.

Se realizó un análisis de los momentos estadísticos de las características del dataset, así como el calculo de la distancia Jensen-Shannon,
histogramas comparativos y la creación, evaluación y análisis de un árbol de decisión.

Se utilizo librerías como numpy, pandas, matplotlib, scipy, pytorch y sklearn para los métodos creados para el trabajo.

\section{Análisis descriptivo de las características en el conjunto}

\end{document}